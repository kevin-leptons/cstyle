\section{Library}

\begin{crules}
    \cmust{Provide one static and one dynamic libarary}
    \cmust{Doesn't provide executable files}
    \cmust{Name library file follow regex} \verb/<pkg-name>[a-z0-9-._]{1,16}/
\end{crules}

Why provide one static and one dynamic library?. Because demand,
other programmers can use one in two when they need. If they use few of parts
of library, static library make small program. However, if they use almost
parts of library, dynamic library is efficient memory and easy to update.
Why provide only one file for each library type?. Because clearly, other
programmers are easy to find that files.

Why doesn't provide executable files?. Because package must be clear, library
package provide "Programming Interface", doen't "User Interface".

<pkg-name> is name of pacakge, follow is suffic specify mete data such as
version, target architectures, build times. \path{_} use for compative with
packge manager's rules. Here are examples:

\begin{center}
    \begin{tabular}{c|c}
            File name & Is valid \\
            \hline
            libc.a & Yes \\
            libcurl.so.4.3.0 & Yes \\
            LibC.a & No
    \end{tabular}
\end{center}

Programming Interface of libraries is parts of source code and it is specify
in Chapter Source code.
